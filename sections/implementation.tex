\chapter{Implementation}

\section{RTP Libraries}
\subsection{oRTP} 
The oRTP library is used in \textit{linphone Open-source VoIP}, which is an open-source VoIP solution created and maintained by Belledonne Communications. The oRTP library is released under the GNU GPLv2 and proprietary license meaning the library can be used for open-source projects and in proprietary solutions.
The oRTP library implements RFC3550 with an API that offers a high as well as low level API. The library supports parsing and composing RTP and RTCP packets. It supports multiple RTP sessions with IPv4 and IPv6. Furthermore, it offers support for different profiles, meaning a custom profile can be implemented.  oRTP has a sparse documentation with only an autogenerated doxygen, where most of the functionality is described. The sourcecode comes with simple examples, that explains some of the library functionality. The library is written in C, meaning language bindings can be utilized to interface the library to scripting languages. The oRTP library can be found in ubuntu + debian's packet repository. At the time of writing, the latest commit on their official github has been made 24 days ago which indicates the project is active.

\subsection{jrtplib}
The jrtplib library is developed at the the Expertise Centre for Digital Media (EDM), a research institute of the Hasselt University. At the time of writing the library has been used in 61 projects listed on \href{http://research.edm.uhasselt.be/jori/cgi-bin/listprojects.py?name=jrtplib}{Project list}. The library is free to use, but must include disclaimer in the source. The library implements RFC3550 and provides primarily a highlevel API, that hides most of the implement details. The library is written in C++, which allows for language bindings. The library is  well documented by giving a thoroughly \textit{Getting Started} and includes 7 examples showing how to utilize the functionality of the library. Unfortunately the author has not done any commits for the past year.

\todo{MIT license}
\todo{Add to design requirements, that software exist from repo + maintained}
\todo{Explain \textit{library properties}}

\begin{table}[h!]
\centering
\begin{tabular}{@{}|l|l|l@{}|}
\hline
\multicolumn{1}{|l|}{\textbf{Library property}} & \multicolumn{1}{|l|}{\textbf{oRTP}}         & \multicolumn{1}{|l|}{\textbf{jrtplib}}       \\ \midrule
\multicolumn{1}{|l|}{In repository}    & \multicolumn{1}{c|}{\checkmark} & \multicolumn{1}{l|}{} \\ \midrule
\multicolumn{1}{|l|}{RTCP impl.} & \multicolumn{1}{c|}{\checkmark} & \multicolumn{1}{c|}{\checkmark} \\ \midrule
\multicolumn{1}{|l|}{Low level API} & \multicolumn{1}{c|}{\checkmark} & \multicolumn{1}{c|}{\checkmark} \\ \midrule
\multicolumn{1}{|l|}{RTP Profile} & \multicolumn{1}{c|}{\checkmark} & \multicolumn{1}{c|}{\checkmark} \\ \midrule
\multicolumn{1}{|l|}{API documented}          & \multicolumn{1}{c|}{\checkmark} & \multicolumn{1}{c|}{\checkmark} \\ \midrule
\multicolumn{1}{|l|}{Library Usage}          & \multicolumn{1}{c|}{\checkmark} & \multicolumn{1}{c|}{\checkmark} \\ \midrule
\multicolumn{1}{|l|}{Actively Maintained}           & \multicolumn{1}{c|}{\checkmark} & \multicolumn{1}{l|}{} \\ \midrule
\multicolumn{1}{|l|}{IPv6 multicast support}           & \multicolumn{1}{c|}{\checkmark} & \multicolumn{1}{c|}{\checkmark} \\ \midrule
\multicolumn{1}{|l|}{Existing perl-binding}           & \multicolumn{1}{c|}{\checkmark} & \multicolumn{1}{c|}{\checkmark} \\ \midrule
\multicolumn{1}{|l|}{Multiple RTP sessions support}           & \multicolumn{1}{c|}{\checkmark} & \multicolumn{1}{c|}{\checkmark} \\ \midrule
\multicolumn{1}{|l|}{Send \& Receive}           & \multicolumn{1}{c|}{\checkmark} & \multicolumn{1}{c|}{\checkmark} \\ \midrule
\multicolumn{1}{|l|}{Includes examples}           & \multicolumn{1}{c|}{\checkmark} & \multicolumn{1}{c|}{\checkmark}  \\ \bottomrule
\end{tabular}
\caption{Comparesion of oRTP and jrtplib}
\label{my-label}
\end{table}


\section{Test}

From wireshark after resample:
4096 = payload
8 bytes = data gram
12 bytes rtp header

4116 bytes i total


\section{Timing in RTCP packets}
\begin{align}
	mon_1 &= Monotonic time \\
	realtime &= now \\
	mon_2 &= Monotonic time \\
	offset &= realtime-(mon_1+mon_2)/2 
\end{align}


\section{Ipv6}
   IPv6 sessions are announced on the address FF0X:0:0:0:0:0:2:7FFE
      where X is the 4-bit scope value.  For example, an announcement
      for a link-local session assigned the address
      FF02:0:0:0:0:0:1234:5678, should be advertised on SAP address
      FF02:0:0:0:0:0:2:7FFE.