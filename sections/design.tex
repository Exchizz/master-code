\chapter{Design} \label{chp:design}

\section{Introduction}
This chapter will go through the design of a \program{Publisher} and \program{Subsriber} that fulfils the requirements listed in table \ref{tab:requirements}.
At first, this chapter will go through the \textit{Design Requirements} in section \ref{sec:design:requirements} which lists the design principles used in this chapter.
In section \ref{sec:design:vcs} a \ac{VCS} is explained, as as a \ac{VCS} has similar functionality, as described in section~\ref{sec:streamingidea}.
From the \ac{VCS}, the most widely used protocols and explained in sections \cref{test, test1, test2}.
At last, the protocols are compared to the requirements from section \ref{sec:analysis:requirements}, in order to find the most suitable protocols.
This chapter will then end up with a list of requirements for the implementation.

\section{Design Requirements} \label{sec:design:requirements}
The proposed system is designed with the concepts of the UNIX philosophy in mind
\begin{itemize}
	\item Programs should do one thing, and do it well
	\item Programs should be able to work together
	\item Programs should be capable of taking streams of text as input
\end{itemize}

\todo{"always use the simplest solution possible" (Rich, 1995, pp. 221 and 231) and McIlroy
recommends (slightly modified) "creating components that do one thing and do it well"}

By complying with the three ideas from above, the system should consist of multiple small programs that can be put together in different ways to serve different purposes.

Since the system is developed for biologists, it should be easy to use and require minimum intervention to get it running. Ideally the system should be “plug and play” such that biologists can take the number of recording boxes required for their application and put it all together without the need to configure software on any of the boxes. \todo{Add Extensibility somewhere, maybe not due to maintainability}

\todo{Add: Write as little code as possible, use already existing code maintained by other developers}
\todo{add: Should be a requirement that it should be easy to write new nodes. Should not require an interface to send commands, should be self-contained as possible.}

\todo{Design inspired by Daniel J. Bernstein, software run in "chains"}
Furthermore the system should also be:

\begin{itemize}
	\item Maintainability
As other people will improve the system in the future, it should also be easy to maintain. This will almost happen automatically if the “unix design rules” are kept in mind during design and development.
\item Modular
The system should consist of multiple small programs so that it can be used in all existing use-cases but also those that might come in the future.
\item Reusability
As much code should be as reusable as possible in order to avoid implementing the same functionality multiple times.
\item Extensible
    By implementing small programs, the system should be easy to extend in the future.
    If functionality is missing in one of the programs, it should be possible to create a new 
    program with the new functionality.
\end{itemize}

However splitting functionality into multiple modules will also be done with caution as it might come with the price of performance. Each time programs are split up, there is usually required some communication between the programs or some exchange of memory. The communication might turn out to be unnecessary performance consumption. Therefore splitting programs up is seen as simplicity(as it eases development and use) vs. performance.







\section{Video conference System}
% http://www.faqs.org/rfcs/rfc5219.html <- mp3 i rtp
% RTP description https://flylib.com/books/en/4.245.1.27/1/
\ac{VCS} utilize audio and video streams, as a video conference is a connection between people residing in separate locations. This connection gives the impression that people participating in the video conference are present in the physical meeting. VCS usually allows for multiple participants to join a meeting where all participants are able to see each other. CVS might allow control of the camera by the remote participant in order to look at the person who is currently speaking. 
Lip-sync is often preferred during the video conference, to further give the impression the remote-participant is present. Lip-sync is when the sound from the remote participants is synchronized with the video stream. This usually has to be implemented by the streaming protocols as sound and video is not necessarily transmitted in the same stream and there by not synchronized when received by the recipient.

\begin{figure}[H]
	\centering
	\includegraphics[width=0.8\textwidth]{figures/vcs_overview.png}
	\caption{Example of a video conference system where five people are participating in a meeting, but the third person is not physically present.} \label{fig:design:vcs}
\end{figure}

\ac{VCS} works similarly to \ac{VoIP}, which is like a conference system between two participants only using sound. A typical protocol stack is shown in figure~\ref{fig:design:protocolstack}. 

rtcp used to adjust codec wrt. bandwidth
\begin{figure}[H]
	\centering
	\includegraphics[width=\textwidth]{figures/protocol_stack}
	\caption{Protocols used in VCS and VoIP. It should be noted, that not protocols necessarily are used at the same time.}
	\label{fig:design:protocolstack}
\end{figure}

Figure \ref{fig:design:protocolstack} shows how the protocol stacks on top of each other, where IP is in the bottom layer in the \textit{Data Plane} and \textit{Control Plane}. The  \textit{Data Plane} is where the encoded audio and video is transferred, between participating nods. As part of the \textit{Data Plane} is also the RTCP protcol, which provides information about the quality between participants. Usually the RTP protocol is used to packetize encoded audio and video. The \textit{Control Plane} is where sessions are initiated. In VoIP, SIP is usually used to initiate sessions, but other protocols such as RTSP and SAP can be used as well. Which protocol is used to initiate the session depends on application and whether it is a session between multicast participants or unicast. SDP is used in SIP to negotiate supported codecs between participants\todo{Between sender and receiver}, but used in RTSP and SAP to describe a session with encoding statically set by the session originator\todo{Correct word?}.\citep{voip_fundamentals}

The protocols in figure \ref{fig:design:protocolstack} are listed in table~\ref{tab:design:protocollist}. Proprietary VCSs such as Skype use closed protocol specifications which are not included in the table.


\begin{table}[H]
	\centering
	\resizebox{\textwidth}{!}{%
		\begin{tabular}{@{}|l|l|l|l|@{}}
			\toprule
			\textbf{Protocol}              & \textbf{Described in} & \textbf{Functionality} \\ \midrule
			Real time Transport Protocol (RTP)    & \ref{sec:design:rtp}                  & Transports video, sound etc.\\ \midrule
			Real time Control Protocol (RTCP)          & \ref{sec:design:rtcp}                & Quality feedback, participant ident. and timing\\ \midrule
			Session Description Protocol (SDP)   & \ref{sec:design:sdp}                  & Describes a RTP session\\ \midrule
			Session Announcement Protocol (SAP)  & \ref{sec:design:sap}                  & Announces a SDP on multicast group\\ \midrule
			Real Time Streaming Protocol (RTSP) & \ref{sec:design:rtsp} & Creates session \\ \bottomrule
		\end{tabular}%
	}
	\caption{Table shows protocols often used in video conference systems}
	\label{tab:design:protocollist}
\end{table}
\todo{Add RTSP and other non-relevant protocols?}

\section{Real-time Transport Protocol} \label{sec:design:rtp}
The RTP protocol is a network protocol for transmitting audio and video data in Voice-over-IP(VoIP), video conferences and online services that require streaming media. RTP is usually run over UDP, however it might be used over TCP as well.
A RTP session consists of one or more participants who are communicating using RTP. A participant may be active in multiple RTP sessions e.g. one session for streaming audio data and another session for streaming video data. For each participant, the session is identified by an IP and port pair to which data should be sent, and a port pair on which data is received. 

The RTP packet is depicted below:

\begin{figure}[h!]
\begin{verbatim}
    0                   1                   2                   3
    0 1 2 3 4 5 6 7 8 9 0 1 2 3 4 5 6 7 8 9 0 1 2 3 4 5 6 7 8 9 0 1
   +-+-+-+-+-+-+-+-+-+-+-+-+-+-+-+-+-+-+-+-+-+-+-+-+-+-+-+-+-+-+-+-+
   |V=2|P|X|  CC   |M|     PT      |       sequence number         |
   +-+-+-+-+-+-+-+-+-+-+-+-+-+-+-+-+-+-+-+-+-+-+-+-+-+-+-+-+-+-+-+-+
   |                           timestamp                           |
   +-+-+-+-+-+-+-+-+-+-+-+-+-+-+-+-+-+-+-+-+-+-+-+-+-+-+-+-+-+-+-+-+
   |           synchronization source (SSRC) identifier            |
   +=+=+=+=+=+=+=+=+=+=+=+=+=+=+=+=+=+=+=+=+=+=+=+=+=+=+=+=+=+=+=+=+
   |            contributing source (CSRC) identifiers             |
   |                             ....                              |
   +-+-+-+-+-+-+-+-+-+-+-+-+-+-+-+-+-+-+-+-+-+-+-+-+-+-+-+-+-+-+-+-+
\end{verbatim}
\caption{Box in an old RFC}
\label{fig:ascii-box}
\end{figure}

many applications
profile
 - static
data types
Timestamp is a 32 bit value.
Id
source
yada..

\subsection{RTP payload}

It is important to be aware of the limits of the RTP protocol specification because it is deliberately incomplete in two ways. First, the standard does not specify algorithms for media playout and timing regeneration, synchronization between media streams, error concealment and correction, or congestion control. These are properly the province of the application designer, and because different applications have different needs, it would be foolish for the standard to mandate a single behavior.
The RTP protocol does not specify how the data should be packet into the payload of an RTP packet. This is defined by profiles and packet types. A profile can contain multiple data formats, and might describe some general details about the content of the RTP packet.
The most used profile is the "RTP/AV audio video for conference with minumal control", which is used for streaming audio and video.
RTP contains packet-type field.

\section{Profile}
since the RTP protocol does not specify how the data should be formated into the payload of the RTP body, this is externally defined by profiles. The most used profile is  the AV-profile, which defines several payload-types for the RTP packets. Which profile is in used is dictated by the SDP \ref{sec:design:sdp}. For historial reasons, only a few data types are defined, however from 96-100 allows for dynamic payloads. If a dynamic payload is used, another standard defines how the data must be packed into the payload of the RTP packet.

The profile dictages general parameter that applies for the RTP session such as:
\begin{itemize}
	\item RTCP interal
	\item Packet types
	\item Clock rate in RTP header
	\item Mappings of parameters to the SDP
\end{itemize}

Dynamic types

\cite{perkins2003rtp}
\section{Real-time Control Protocol} \label{sec:design:rtcp}
RTCP is used to provide reception quality feedback, participant identification, and synchronization between media streams. RTCP runs alongside RTP and provides periodic reporting of this information. Although data packets are typically sent every few milliseconds , the control protocol operates on the scale of seconds. The information sent in RTCP is necessary for synchronization between media streams ”for example, for lip synchronization between audio and video ”and can be useful for adapting the transmission according to reception quality feedback, and for identifying the participants . 

receiver report (RR), sender report (SR), source description (SDES), membership management (BYE), and application-defined (APP). 

\section{Session Description Protocol} \label{sec:design:sdp}
The Session Description Protocol(SDP) is a protocol for describing RTP streams.
SDP does not provide any media nor specify how the SDP must be transferred.
SDP is widely used in VoIP and conference systems, where parameters must be known to receivers before the RTP streams can be decoded and presented. In multicast systems, SDP is also used to announce streams such that receivers know which multicast group to join in order to get the streams. Usually the SDP is sent from the originator of the session, but SDP can also be used to negotiated parameters between originator and participants. In general the SDP most provide enough information about a stream such that a participant can decide whether the stream should be joined or not.  SDP is designed to be extensible to support new media types and formats. If a dynamic packet-type is chosen in the audio/video profile, SDP supports adding additional information to the session description. 

An SDP file comprises of strictly defined key-value pairs. The SDP describes several keys where some must be provided and others are optional. A key-value pair is defined as shown below:
\begin{verbatim}
<key>=<value>
\end{verbatim}

No spaces are allowed around the key or value.
The order of the keys are dictated by the RFC. The reason for the strict format is to ease parsing and to easily detect errors in a SDP file. An example of a SDP is shown below:

\begin{verbatim}
    v=0
    o=jdoe 2890844526 2890842807 IN IP4 10.47.16.5
    s=SDP Seminar
    i=A Seminar on the session description protocol
    u=http://www.example.com/seminars/sdp.pdf
    e=j.doe@example.com (Jane Doe)
    c=IN IP4 224.2.17.12/127
    t=2873397496 2873404696
    a=recvonly
    m=audio 49170 RTP/AVP 96
    a=rtpmap:96 L8/8000
\end{verbatim}
The keys used above are described below in the list below. \\
\textbf{v=} Version. Currently only version 0 exists. \\
\textbf{o=} Originator, information about who sends the streams.Source IP, sessionid etc. \\
\textbf{s=} Session name. \\
\textbf{i=} Session Information \\
\textbf{u=} Url to more information about the session.\\
\textbf{e=} Email address of originator.\\
\textbf{c=} Connection data comprising of "nettype" "addrtype" "connection-address".
\begin{itemize}
	\item The first field "nettype" is the network type where only Internet is defined. 
	\item The second field "addrtype" is the type of the address. This can either be IPv4 or IPv6.
	\item The third field "connection-address" defines the address participants must connect to in unicast applications and what group participants must join when multicast is used.
\end{itemize}
\textbf{t=} - Timing defines when the stream starts and stops. \\
\textbf{a=} - Attribute used to extend the SDP to support additional properties. Recvonly tells participants to only receive from this session. \\
\textbf{m=} - Media comprising of "media" "port" "proto" "fmt" ...
\begin{itemize}
	\item Media defines whether the stream is audio, video, text or application.
	\item Port is where participants must connect to when unicast is used. If multicast, this port is where participants will receive the stream.
	\item Proto is the transport protocol. This is usually RTP/AVP, which means RTP is used with the audio/video profile.
	\item Fmt is application specific. If audio/video profile is chosen, fmt describes the payload-type used in the RTP stream.
\end{itemize}
\textbf{a=} is an additional parameter. rtpmap is used when a dynamic payload type is used in the audio/video profile. rtpmap maps the packet-type to the encoding used of the RTP payload. In the example above, L8 is defined as 8 bit raw sound with sample frequency of 8 khz. Additional arguments can be set by adding a /<param> to the list, such as /2 for stereo.

All key-value pairs above m= applies to all session, where key-value pairs below a m= only applies to that particular m=.

The listed key-value pairs is not exhaustive. SDP supports more keys as described in RFC4566.

%A media is defined as:
%            m=audio 49232 RTP/AVP 0
%            m=audio 1337 RTP/AVP 98
%When a dynamic packet type is in use, the following parameter is given:
%            a=rtpmap:98 L16/16000/2
%This associates the the dynamic payload type, 9, with the encoded format which in this case is L16 with a clock frequency of 16000 stereo.

\section{Session Announcement Protocol} \label{sec:design:sap}
\section{Real Time Streaming Protocol} \label{sec:design:rtsp}

\section{Gstreamer}
Gstreaner is a open source tool available, that implements some of the protocols described in section \ref{sec:design:protocols}.
\todo{Describe gstreamer + pros and cons}
Cons:
 - 64bit timestamp injected
Gstreamer is a pipeline based streaming framework that aims to make it easy to work with streams. Gstreamer is plugin based which allows for adding functionality as needed in applications. The idea of plugins is to create plugins are have a well defined responsibility such as reading a file, encoding, playing etc. The plugins are usually connected in gstreamer pipelines, however then can be used in independent applications. Gstreamer plugins are connected using pads, where each plugin can have a source and sink pad. A source pad can then be connected to the next plugin's sink pad. An example of plugins in a pipelines is showed below:
\begin{figure}
	\includegraphics[width=1\textwidth]{figures/bin-element-ghost.png}
\end{figure}

\todo{Describe example figure + rtpbin + rtppay/depay}


\todo{Compare gstreamer with custom impl.}


\subsection{Publisher \& Subscriber}
This section presents the design of the \program{Publisher} and \program{Subscriber} from the extracted requirements from the analysis chapter. The end of this section compares the implementation requirements with the protocols from section \ref{sec:design:protocols}.

\subsubsection{Publisher}

\subsubsection{Subscriber}

Describe given the nature of the stream, that publisher/subscriber most react on incoming data, and else do nothing until a new packet arrives.

This is similar to graphical interfaces, where the input is key press or clicks. This is event driven meaning a callback is invoked when an event is happening.
The subscriber should listen for the following events:
\begin{itemize}
	\item Incoming data packet from the producer
	\item Incoming metadata packet from the producer
	\item Periodically when metadata must be sent
\end{itemize}
\todo{Call "periodically event" for "temporal event"?}

The publisher should listen for the following events:
\begin{itemize}
	\item Incoming packet on stream
	\item Incoming metadata packet on stream
	\item Periodically when metadata must be sent.
\end{itemize}

alternative to event is polling, where the subscriber/publisher polls for data. This would be very CPU consuming and not utilize the functionality provided by the kernel.


\section{Historian}

Comparison of rtpdump(rtptools) and tcpdump+tcpreplay.

% Please add the following required packages to your document preamble:
% \usepackage{graphicx}
\begin{table}[H]
\centering
\resizebox{\textwidth}{!}{%
\begin{tabular}{|l|c|c|}
\hline
\textbf{}       & \textbf{tcpdump/tcpreplay} & \textbf{rtpdump/rtpplay} \\ \hline
Record duration &                            &                          \\ \hline
IPv6 support    & \checkmark  & X                        \\ \hline
Multiple replay &                            &                          \\ \hline
Replay of RTCP  &          \checkmark                 &         X                 \\ \hline
\end{tabular}%
}
\caption{My caption}
\label{my-label}
\end{table}
\section{Stream Topology}

Master vs. nomaster
Cost of wellknown information(Profile + address, port) at all ends.
Describe how streams are put into sessions(RFC about semantics and taxonomics, pros and cons)

