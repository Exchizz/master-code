\chapter{Design}

% http://www.faqs.org/rfcs/rfc5219.html <- mp3 i rtp
% RTP description https://flylib.com/books/en/4.245.1.27/1/
Video conference system(VCS) utilize audio and video streams, as a video conference is a connection between people residing in separate locations. This connection gives the impression that people participating in the video conference are present in the physical meeting. VCS usually allows for multiple participants to join a meeting where all participants are able to see each other. CVS might allow control of the camera by the remote participant in order to look at the person who is currently speaking. 
Lip-sync is often preferred during the video conference, to further give the impression the remote-participant is present. Lip-sync is when the sound from the remote participants is synchronized with the video stream. This usually has to be implemented by the streaming protocols as sound and video is not necessarily transmitted in the same stream and there by not synchronized when received by the recipient.

\missingfigure{sketch of video conference}

VCS usually utilizes the protocols shown in table ~\ref{table:vcs:protocols}. Proprietary VCSs such as Skype use closed protocol specifications which is not included in the table.


\begin{table}[]
	\centering
	\resizebox{\textwidth}{!}{%
		\begin{tabular}{@{}|l|l|l|l|@{}}
			\toprule
			\textbf{Protocol}             & \textbf{Abreviation} & \textbf{Described in} & \textbf{Note} \\ \midrule
			Realtime Transport Protocol   & RTP                  &                       & Transports video, sound etc.\\ \midrule
			RTP Control Protocol          & RTCP                 &                       & Quality feedback, participant identification and timing\\ \midrule
			Session Description Protocol  & SDP                  &                       & Describes a RTP session\\ \midrule
			Session Announcement Protocol & SAP                  &                       & Announces a SDP on multicast group\\ \bottomrule
		\end{tabular}%
	}
	\caption{Table showing protocols usually used in a video conference system.}
	\label{my-label}
\end{table}
\todo{Add RTSP and other non-relevant protocols?}

\section{Real-time Transport Protocol}
The RTP protocol is a network protocol for transmitting audio and video data in Voice-over-IP(VoIP), video conferences and online services that require streaming media. RTP is usually run over UDP, however it might be used over TCP as well.
A RTP session consists of one or more participants who are communicating using RTP. A participant may be active in multiple RTP sessions e.g. one session for streaming audio data and another session for streaming video data. For each participant, the session is identified by an IP and port pair to which data should be sent, and a port pair on which data is received. 

The RTP packet is depicted below:

\begin{figure}[h!]
\begin{verbatim}
    0                   1                   2                   3
    0 1 2 3 4 5 6 7 8 9 0 1 2 3 4 5 6 7 8 9 0 1 2 3 4 5 6 7 8 9 0 1
   +-+-+-+-+-+-+-+-+-+-+-+-+-+-+-+-+-+-+-+-+-+-+-+-+-+-+-+-+-+-+-+-+
   |V=2|P|X|  CC   |M|     PT      |       sequence number         |
   +-+-+-+-+-+-+-+-+-+-+-+-+-+-+-+-+-+-+-+-+-+-+-+-+-+-+-+-+-+-+-+-+
   |                           timestamp                           |
   +-+-+-+-+-+-+-+-+-+-+-+-+-+-+-+-+-+-+-+-+-+-+-+-+-+-+-+-+-+-+-+-+
   |           synchronization source (SSRC) identifier            |
   +=+=+=+=+=+=+=+=+=+=+=+=+=+=+=+=+=+=+=+=+=+=+=+=+=+=+=+=+=+=+=+=+
   |            contributing source (CSRC) identifiers             |
   |                             ....                              |
   +-+-+-+-+-+-+-+-+-+-+-+-+-+-+-+-+-+-+-+-+-+-+-+-+-+-+-+-+-+-+-+-+
\end{verbatim}
\caption{Box in an old RFC}
\label{fig:ascii-box}
\end{figure}

many applications
profile
 - static
data types
Timestamp is a 32 bit value.
Id
source
yada..

\subsection{RTP payload}

It is important to be aware of the limits of the RTP protocol specification because it is deliberately incomplete in two ways. First, the standard does not specify algorithms for media playout and timing regeneration, synchronization between media streams, error concealment and correction, or congestion control. These are properly the province of the application designer, and because different applications have different needs, it would be foolish for the standard to mandate a single behavior.
The RTP protocol does not specify how the data should be packet into the payload of an RTP packet. This is defined by profiles and packet types. A profile can contain multiple data formats, and might describe some general details about the content of the RTP packet.
The most used profile is the "RTP/AV audio video for conference with minumal control", which is used for streaming audio and video.
RTP contains packet-type field.

\cite{perkins2003rtp}
\section{Real-time Control Protocol}
RTCP is used to provide reception quality feedback, participant identification, and synchronization between media streams. RTCP runs alongside RTP and provides periodic reporting of this information. Although data packets are typically sent every few milliseconds , the control protocol operates on the scale of seconds. The information sent in RTCP is necessary for synchronization between media streams ”for example, for lip synchronization between audio and video ”and can be useful for adapting the transmission according to reception quality feedback, and for identifying the participants . 

receiver report (RR), sender report (SR), source description (SDES), membership management (BYE), and application-defined (APP). 

\section{Session Description Protocol}
\todo{Description of SDP}

\subsection{Session Announcement Protocol}
\todo{Description of SAP}

This will also be useful in use case \ref{usecase:verifyWorking}, as this provides a way for the system to collect information about which batboxes are successfully connected to the system.

\section{Gstreamer}
\todo{Describe gstreamer + pros and cons}
Cons:
 - 64bit timestamp injected
 


\section{RTP Libraries}
\subsection{oRTP} 
The oRTP library is used in \textit{linphone Open-source VoIP}, which is an open-source VoIP solution created and maintained by Belledonne Communications. The oRTP library is released under the GNU GPLv2 and proprietary license meaning the library can be used for open-source projects and in proprietary solutions.
The oRTP library implements RFC3550 with an API that offers a high as well as low level API. The library supports parsing and composing RTP and RTCP packets. It supports multiple RTP sessions with IPv4 and IPv6. Furthermore, it offers support for different profiles, meaning a custom profile can be implemented.  oRTP has a sparse documentation with only an autogenerated doxygen, where most of the functionality is described. The sourcecode comes with simple examples, that explains some of the library functionality. The library is written in C, meaning language bindings can be utilized to interface the library to scripting languages. The oRTP library can be found in ubuntu + debian's packet repository. At the time of writing, the latest commit on their official github has been made 24 days ago which indicates the project is active.

\subsection{jrtplib}
The jrtplib library is developed at the the Expertise Centre for Digital Media (EDM), a research institute of the Hasselt University. At the time of writing the library has been used in 61 projects listed on \href{http://research.edm.uhasselt.be/jori/cgi-bin/listprojects.py?name=jrtplib}{Project list}. The library is free to use, but must include disclaimer in the source. The library implements RFC3550 and provides primarily a highlevel API, that hides most of the implement details. The library is written in C++, which allows for language bindings. The library is  well documented by giving a thoroughly \textit{Getting Started} and includes 7 examples showing how to utilize the functionality of the library. Unfortunately the author has not done any commits for the past year.


\begin{table}[h!]
\centering
\begin{tabular}{@{}|l|l|l@{}|}
\hline
\multicolumn{1}{|l|}{\textbf{Thang}} & \multicolumn{1}{|l|}{\textbf{oLIB}}         & \multicolumn{1}{|l|}{\textbf{jrtplib}}       \\ \midrule
\multicolumn{1}{|l|}{In repo}    & \multicolumn{1}{l|}{} & \multicolumn{1}{l|}{} \\ \midrule
\multicolumn{1}{|l|}{RTCP impl.} & \multicolumn{1}{l|}{} & \multicolumn{1}{l|}{} \\ \midrule
\multicolumn{1}{|l|}{Documentation available}          & \multicolumn{1}{l|}{} & \multicolumn{1}{l|}{} \\ \midrule
\multicolumn{1}{|l|}{Actively Maintained}           & \multicolumn{1}{l|}{} & \multicolumn{1}{l|}{} \\ \midrule
\multicolumn{1}{|l|}{IPv6 support}           & \multicolumn{1}{l|}{} & \multicolumn{1}{l|}{} \\ \midrule
\multicolumn{1}{|l|}{Includes examples}           & \multicolumn{1}{l|}{} & \multicolumn{1}{l|}{} \\ \midrule
\multicolumn{1}{|l|}{License}           & \multicolumn{1}{l|}{} & \multicolumn{1}{l|}{} \\ \bottomrule
\end{tabular}
\caption{My caption}
\label{my-label}
\end{table}
